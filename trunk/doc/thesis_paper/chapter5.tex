\chapter{Conclusion}
\label{chap:conclusion}
Throughout this paper we have shown a variety of results for improving the visualization and simulation of both mobile ad hoc and cognitive networks.  In Chapter~\ref{chap:network_vis} we showed two visualization techniques useful for simulated mobile ad hoc networks.  We first presented a physical layer visualization of signal radiation patterns along with regions of acceptable SINR.  This visualization helped to improve the understanding of resource allocation and parameter selection on network design.  Our second visualization introduced the use of digital terrain in visualizing simulated mobile ad hoc network layouts.  The inclusion of terrain in the visualizations adds a level of realism and a stronger understanding of the affects that terrain can have on the performance of mobile ad hoc networks.

The work presented in Chapter~\ref{chap:graph_simp} provided interesting results on our algorithm for visualizing the structure of complex graphs.  Our algorithm showed that the simplification of the graph provides an approach to visualizing the fundamental structure of the graph by displaying the most important nodes, where importance may be based on the topology of the graph or external factors.  The use of weighting metrics to isolate different types of structures proved to be an effective means of visualizing a variety of complex graphs.

The final section of work discussed in Chapter~\ref{chap:cognet_gpu}, provided some very promising results on improving the performance of cognitive network spectrum sensing algorithms.  We selected two well-known algorithms for spectrum sensing, periodogram based energy detection and cyclostationary spectral analysis via the FFT Accumulation Method.  We were able to show that porting these spectrum sensing algorithms onto the GPU computing architecture yielded significant performance improvement, up to 30x speedup over CPU implementations.
