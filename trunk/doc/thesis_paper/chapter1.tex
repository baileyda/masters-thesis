\chapter{Introduction}

The field of computer networking is one in constant evolution.  New advanced network architectures are developed in theory but are hard to directly implement.  Simulation of these advanced networks provides a powerful tool to help develop the implementation of a network architecture while also studying the results of the implementation.  Visualization of the simulated network information can play a big role in our understanding of the simulation results and the networks themselves.  A creative visual display of information can increase our understanding of a simulation by providing an easier means of analysis.  Research into visualizing various data associated with the simulation of mobile ad hoc networks is presented in Chapter~\ref{chap:network_vis}.  Here we discuss two visualization techniques, one for visualizing network resources at a particular layer of the network, and another for providing a richer visualization of networks by introducing the use of digital terrain.  We discuss the results of implementing these visualizations in a mobile ad hoc network simulator called OMAN.

Another approach to visualizing complex information is taken in Chapter~\ref{chap:graph_simp}.  This chapter focuses on developing an algorithm to help visualize the structure of large complex graphs.  Large complex graphs tend to have very cluttered and unintelligible visualizations simply due to the large amount information needed to be presented.  Our algorithm focuses on simplifying these graphs to some basic underlying structure and then presenting this in visualization.  We use a variety of different metrics for capturing and quantifying different features of the graph to visualize.  This capability allows a user to interactively discover key nodes in the graph and examine their interconnectivity.  We present the results of applying our algorithm to a variety of different types of complex graphs.

The visualization of data is useful, but we also provide work done on the improvement of the simulations.  Chapter~\ref{chap:cognet_gpu} provides an investigation into how we can potentially improve the computational performance of different algorithms for cognitive networking.  Specifically, we explore the improvement of several well known and well defined spectrum sensing algorithms.  Spectrum sensing is an important step for the proper operation of cognitive networks.  These algorithms need to perform with as little latency as possible in order to provide the best experience for all users of the cognitive network.  We explore the benefits of adapting spectrum sensing algorithms to a programmable Graphics Processing Unit (GPU) in order to significantly reduce their computation times.  We present a comparison of execution times on a standard CPU based implementation versus those executing on a GPU based implementation. 
